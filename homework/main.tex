% !TeX spellcheck = en_US
\documentclass[12pt,letterpaper]{article}
\usepackage{fullpage}
\usepackage[top=2cm, bottom=4.5cm, left=2.5cm, right=2.5cm]{geometry}
\usepackage{amsmath,amsthm,amsfonts,amssymb,amscd}
\usepackage{lastpage}
\usepackage{enumerate}
\usepackage{fancyhdr}
\usepackage{mathrsfs}
\usepackage{xcolor}
\usepackage{graphicx}
\usepackage{listings}
\usepackage{hyperref}
\usepackage{listings}
\usepackage{xcolor}
 
\definecolor{codegreen}{rgb}{0,0.6,0}
\definecolor{codegray}{rgb}{0.5,0.5,0.5}
\definecolor{codepurple}{rgb}{0.58,0,0.82}
\definecolor{backcolour}{rgb}{0.95,0.95,0.92}
 
\lstdefinestyle{mystyle}{
    backgroundcolor=\color{backcolour},   
    commentstyle=\color{codegreen},
    keywordstyle=\color{magenta},
    numberstyle=\tiny\color{codegray},
    stringstyle=\color{codepurple},
    basicstyle=\ttfamily\footnotesize,
    breakatwhitespace=false,         
    breaklines=true,                 
    captionpos=b,                    
    keepspaces=true,                 
    numbers=left,                    
    numbersep=5pt,                  
    showspaces=false,                
    showstringspaces=false,
    showtabs=false,                  
    tabsize=2
}
 
\lstset{style=mystyle}

\newcommand{\norm}[1]{\lVert#1\rVert_2}

\hypersetup{%
  colorlinks=true,
  linkcolor=blue,
  linkbordercolor={0 0 1}
}
 
\renewcommand\lstlistingname{Algorithm}
\renewcommand\lstlistlistingname{Algorithms}
\def\lstlistingautorefname{Alg.}

\lstdefinestyle{Python}{
    language        = Python,
    frame           = lines, 
    basicstyle      = \footnotesize,
    keywordstyle    = \color{blue},
    stringstyle     = \color{green},
    commentstyle    = \color{red}\ttfamily
}

\setlength{\parindent}{0.0in}
\setlength{\parskip}{0.05in}

% Edit these as appropriate
\newcommand\course{SDSC 8005}
\newcommand\hwnumber{1}                  % <-- homework number
\newcommand\NetIDa{CHANG Haoliang}           % <-- NetID of person #1
\newcommand\NetIDb{55197643}           % <-- NetID of person #2 (Comment this line out for problem sets)

\pagestyle{fancyplain}
\headheight 35pt
\lhead{\NetIDa}
\lhead{\NetIDa\\\NetIDb}                 % <-- Comment this line out for problem sets (make sure you are person #1)
\chead{\textbf{\Large Homework \hwnumber}}
\rhead{\course \\ \today}
\lfoot{}
\cfoot{}
\rfoot{\small\thepage}
\headsep 1.5em

\begin{document}

\section*{Question 1}

Answer to the Question 1 goes here.

\subsection*{(a)}

Since one can produce and sell a fraction of laptop, we could formulate this problem as the following:

\begin{equation}
\begin{aligned}
& \underset{x_1, x_2}{\text{minimize}}
& & -1200x_1 - 1300x_2 \\
& \text{s.t.} & &  2x_1+x_2 \leq 500 \\
& & & 2x_1+3x_2 \leq 800 \\
& & &  0 \leq x_1 \leq 220 \\
& & &  0 \leq x_2 \leq 180 \\
\end{aligned}
\label{optimization1_1}
\end{equation}

where $x_1, x_2$ represent the number of full-size laptops and compact laptops produced respectively. 

\subsection*{(b)}

Using the Python package \textbf{cvxpy}, we write the following codes to find the optimal solution:

\begin{lstlisting}[language=Python, caption=Question 1.b]
import cvxpy as cp

x_1 = cp.Variable()
x_2 = cp.Variable()

constraint = []
constraint.append(2*x_1 + x_2 <= 500)
constraint.append(2*x_1 + 3*x_2 <= 800)
constraint.append(x_1<=220)
constraint.append(x_2<=180)
constraint.append(x_1>=0)
constraint.append(x_2>=0)

prob = cp.Problem(cp.Maximize(1200*x_1 + 1300*x_2), constraint)
prob.solve()

print('The optimal value is: {}'.format(prob.value))
print('The optimal value for x_1 is: {}'.format(x_1.value))
print('The optimal value for x_2 is: {}'.format(x_2.value))
\end{lstlisting}

Based on the above codes, the optimal solution of the optimization problem in \ref{optimization1_1} is: $x_1=175, x_2=150$. The objective value is 405000. 

\subsection*{(c)}

By adding 10 hours in general assembly process, the optimization problem given in \ref{optimization1_1} could be reformatted as follows:

\begin{equation}
\begin{aligned}
& \underset{x_1, x_2, r}{\text{minimize}}
& & -1200x_1 - 1300x_2\\
& \text{s.t.} & &  2x_1+x_2 \leq 510 \\
& & & 2x_1+3x_2 \leq 800 \\
& & &  0 \leq x_1 \leq 220 \\
& & &  0 \leq x_2 \leq 180 \\
\end{aligned}
\label{optimization1_3}
\end{equation}

Then using the following codes:

\begin{lstlisting}[language=Python, caption=Question 1.c]
import cvxpy as cp

# Construct two variables
x_1 = cp.Variable()
x_2 = cp.Variable()

# Construct constraints
constraint = []
constraint.append(2*x_1 + x_2 <= 510) # add 10 hours in here
constraint.append(2*x_1 + 3*x_2 <= 800)
constraint.append(x_1<=220)
constraint.append(x_2<=180)
constraint.append(x_1>=0)
constraint.append(x_2>=0)

# Solve the problem
prob = cp.Problem(cp.Maximize(1200*x_1 + 1300*x_2), constraint)
prob.solve()
# Get the result
print('The optimal value is: {}'.format(prob.value))
print('The optimal value for x_1 is: {}'.format(x_1.value))
print('The optimal value for x_2 is: {}'.format(x_2.value))
\end{lstlisting}

we could get the optimal solution is: $x_1 = 182.5, x_2 = 145$. Based on this setting, Junhui could give Clint at most $\$$2500, in other words, $\$$250 per hour.

\section*{Question 2}

Answer to the Question 2 goes here.

\subsection*{(a)}

The problem could be formulated as the following:

\begin{equation}
\begin{aligned}
& \underset{x_1, x_2, x_3, x_4, x_5, x_6}{\text{minimize}}
& & -7000x_1 - 5000x_2 - 6000x_3 - 7000x_4 - 5000x_5 - 6000x_6 \\
& \text{s.t.} & &  3x_1+2x_2+3x_3 \leq 30 \\
& & & 4x_4+3x_5+2x_6 \leq 20 \\
& & &  x_1 + x_4 \leq 10 \\
& & &  x_2 + x_5 \leq 8 \\
& & &  x_3 + x_6 \leq 7 \\
& & &  x_1, x_2, x_3, x_4, x_5, x_6 \in Z_{\geq 0} \\
\end{aligned}
\label{optimization2_1}
\end{equation}

Where $x_1, x_3$ denote the number of watch A produced by factory 1 and factory 2, $x_2, x_4$ represent the number of watch B produced by factory 1 and factory 2 and $x_3, x_6$ denote the number of watch C produced by factory 1 and factory 2.

Using Python \textbf{cvxopt} package, we could write the following codes:

\begin{lstlisting}[language=Python, caption=Question 2.a]
import numpy as np
from cvxopt import matrix, solvers
from cvxopt.glpk import ilp

# The coefficients of inequalities
A = matrix([ [1.0, 0.0, 0.0, 3.0, 0.0, -1.0, 0.0, 0.0, 0.0, 0.0, 0.0], 
            [1.0, 0.0, 0.0, 2.0, 0.0, 0.0, -1.0, 0.0, 0.0, 0.0, 0.0], 
           [0.0, 1.0, 0.0, 3.0, 0.0, 0.0, 0.0, -1.0, 0.0, 0.0, 0.0], 
           [0.0, 1.0, 0.0, 0.0, 4.0, 0.0, 0.0, 0.0, -1.0, 0.0, 0.0],
           [0.0, 0.0, 1.0, 0.0, 3.0, 0.0, 0.0, 0.0, 0.0, -1.0, 0.0], 
           [0.0, 0.0, 1.0, 0.0, 2.0, 0.0, 0.0, 0.0, 0.0, 0.0, -1.0] ])

# The RHS values of inequalities
b = matrix([ 10.0, 8.0, 7.0, 30.0, 20.0, 0.0, 0.0, 0.0,0.0, 0.0, 0.0 ])
# The coefficients of the objective function
c = matrix([ -7000.0, -7000.0, -5000.0, -5000.0, -6000.0, -6000.0 ])

I=set(range(0, 6)) # set of indices of integer variables
B=set() # set of indices of binary variables

# Get optimial solution
(status,x)=ilp(c,A,b, matrix(1., (0,6)),matrix(1., (0,1)),I,B)
# print(status) # check the solution status
print(x)
\end{lstlisting}

Finally, the optimal solution of the optimization problem \ref{optimization2_1} is $x_1 = 0, x_2 = 1, x_3 = 3, x_4 = 1, x_5 = 0, x_6 = 7$.

\subsection*{(b)}

Here, we introduce a big number $M$ and formulate the problem as the following:

\begin{equation}
\begin{aligned}
& \underset{x_1, x_2, x_3, x_4, x_5, x_6}{\text{minimize}}
& & -7000x_1 - 5000x_2 - 6000x_3 - 7000x_4 - 5000x_5 - 6000x_6 \\
& \text{s.t.} & &  3x_1+2x_2+3x_3 \leq 30 \\
& & & 4x_4+3x_5+2x_6 \leq 20 \\
& & &  x_1 + x_4 \leq 10 \\
& & &  x_2 + x_5 \leq 8 \\
& & &  x_3 + x_6 \leq 7 \\
& & &  y_1 + y_2 + y_3 \leq 2 \\
& & &  y_4 + y_5 + y_6 \leq 2 \\
& & &  8 - x_1 -x_2 \leq M(1-y_1) \\
& & &  x_1 \leq My_1 \\
& & &  x_2 \leq My_2 \\
& & &  x_3 \leq My_3 \\
& & &  x_4 \leq My_4 \\
& & &  x_5 \leq My_5 \\
& & &  x_6 \leq My_6 \\
& & &  x_1, x_2, x_3, x_4, x_5, x_6 \in Z_{\geq 0} \\
& & &  y_1, y_2, y_3, y_4, y_5, y_6 \in \{0, 1\} \\
\end{aligned}
\label{optimization2_2}
\end{equation}

Here, $y_1, y_4$ denote whether factory 1 and factory 2 produce watch A or not. $y_2, y_5$ represent whether factory 1 and factory 2 produce watch B or not. Finally, $y_3, y_6$  mean factory 1 and factory 2 produce watch C or not.

Based on the above settings, we use the following codes to find the optimal solution:

\begin{lstlisting}[language=Python, caption=Question 2.b]
import cvxpy as cp

# Create variables.
x_1 = cp.Variable(integer=True)
x_2 = cp.Variable(integer=True)
x_3 = cp.Variable(integer=True)
x_4 = cp.Variable(integer=True)
x_5 = cp.Variable(integer=True)
x_6 = cp.Variable(integer=True)

y_1 = cp.Variable(boolean=True)
y_2 = cp.Variable(boolean=True)
y_3 = cp.Variable(boolean=True)
y_4 = cp.Variable(boolean=True)
y_5 = cp.Variable(boolean=True)
y_6 = cp.Variable(boolean=True)

# Set a big number M
M = 100000000

# Create constraints.
constraints = [3*x_1 + 2*x_2 + 3*x_3 <= 30,
4*x_4 + 3*x_5 + 2*x_6 <= 20, 
x_1 + x_4 <= 10, 
x_2 + x_5 <= 8, 
x_3 + x_6 <= 7,
y_1 + y_2 + y_3 <= 2, 
y_4 + y_5 + y_6 <= 2, 
8 - x_1 - x_2 <= M*(1 - y_1), 
x_1 <= M*y_1,
x_1 >= 0,
x_2 <= M*y_2,
x_2 >= 0,
x_3 <= M*y_3,
x_3 >= 0,
x_4 <= M*y_4,
x_4 >= 0,
x_5 <= M*y_5,
x_5 >= 0,
x_6 <= M*y_6, 
x_6 >= 0]

# Form objective.
obj = cp.Minimize(-7000*x_1 - 5000*x_2 - 6000*x_3 -7000*x_4 - 5000*x_5 - 6000*x_6 )

# Form and solve problem.
prob = cp.Problem(obj, constraints)
prob.solve()
print("The values of Xs are: x_1: {}, x_2: {}, x_3: {}, x_4: {}, x_5: {}, x_6: {}".format(x_1.value, x_2.value, x_3.value, x_4.value, x_5.value, x_6.value))
\end{lstlisting}

The optimal solution is $x_1 = 6, x_2 = 6, x_3 = 0, x_4 = 0, x_5 = 2, x_6 = 7$. The objective value is 124000.

\end{document}
